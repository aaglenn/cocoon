\documentclass[letterpaper,10pt,twocolumn]{article}
\usepackage[pdftex]{graphicx}
\usepackage{url}
\usepackage{cite}
\usepackage{epigraph}
\def\docstatus{}
\usepackage{titlesec}
\usepackage{authblk}
\usepackage{natbib}

\usepackage[letterpaper, left=0.7in, right=0.7in, top=1.5in, 
bottom=1in]{geometry}

\usepackage{authblk}
\makeatletter
\renewcommand\AB@affilsepx{ \quad \protect\Affilfont}
\makeatother


\setkeys{Gin}{keepaspectratio=true,clip=true,draft=false,width=\linewidth}
\usepackage{times,cite,url}
\renewcommand{\ttdefault}{cmtt}	% CM rather than courier for \tt


\newcommand{\Comment}[1]{\relax}

\newcommand{\leonid}[1]{\Comment{#1 [leonid]}}
\newcommand{\code}[1]{\texttt{#1}}

\titlespacing*{\paragraph} {0pt}{.5ex plus .2ex minus .2ex}{1em}

\titleformat*{\section}{\bfseries}
\titlespacing*{\section} {0pt}{1ex plus .2ex minus .2ex}{0.3em}


\pagenumbering{gobble}

\begin{document}

\title{\vspace{-3cm}Towards Correct-by-Construction SDN}

\author[1]{Leonid Ryzhyk}
\author[2]{Nikolaj Bj{\o}rner}
\author[3]{Marco Canini}
\author[1]{Jean-Baptiste Jeannin}
\author[1]{Nina Narodytska}
\author[1]{Cole Schlesinger}
\author[1]{Douglas B. Terry}
\author[2]{George Varghese}

\affil[1]{Samsung Research America}
\affil[2]{Microsoft Research}
\affil[3]{Universit\'e catholique de Louvain}

\date{}
\maketitle

High-level SDN languages raise the level of abstraction in SDN 
programming from managing individual switches to programming 
network-wide policies.  In this talk, we present Cocoon (for 
Correct by Construction Networking), a language designed around 
the idea of \emph{iterative refinement}.  The programmer starts 
with a high-level functional description of the desired network 
behavior, focusing on the service the network should provide to 
each packet, as opposed to how this service is implemented within 
the network fabric.  The developer then iteratively refines the 
top-level specification, adding details of the topology, routing, 
fault recovery, etc., until reaching a level of detail sufficient 
for the Cocoon compiler to generate an SDN application that 
manages network switches via the southbound interface (we 
currently support P4~\cite{Bosshart_DGIMRSTVVW_14}).  We designed 
Cocoon with the following goals in mind:

\paragraph{Correctness} Cocoon uses the Corral model 
checker~\cite{Lal_QL_12} to establish that each refinement 
correctly implements the behavior it refines, ensuring that 
behaviors specified at any refinement step hold on the resulting 
SDN application.

%Crucially, we aim to provide strong correctness guarantees through 
%formal verification for complex dynamic networks.  Examples of 
%such networks that motivate this work are: software-defined 
%cellular cores~\cite{Jin_LVR_13}, cloud data centers, 
%WANs~\cite{Jain_KMOPSVWZZZHSV_13}, etc.

\paragraph{Generality} Cocoon enables a wide range of SDN 
applications, ranging from network virtualization, through 
software-defined IXPs, to home networks. 

%This is in contrast to many existing languages designed for a 
%particular application domain, such as service 
%chaining~\cite{Hinrichs_GCMS_09} or virtualization~\cite{neutron}.

\paragraph{Dynamism} A Cocoon program specifies both data and 
control plane behavior, akin to languages like 
FlowLog~\cite{Nelson_FSK_14}, Maple~\cite{Voellmy_WYFH_13}, and 
VeriCon~\cite{Ball_BGIKSSV_14}.  This is in contrast to languages 
such as NetKAT~\cite{Anderson_FGJKSW_14}, which specify a snapshot 
of data plane behavior but rely on a general-purpose programming 
language to implement the control plane by emitting a stream of 
snapshots in response to network events.

%A Cocoon program captures both data and control plane behavior.  
%The Cocoon interpreter runs as an SDN application on top of the 
%controller, dynamically managing the network configuration 
%according to the Cocoon program.  This is in contrast to languages 
%such as NetKAT~\cite{Smolka_EFG_15} that specify an instant 
%snapshot of the dataplane configuration.  The SDN application 
%manages the network by generating updates to the NetKAT program; 
%the semantics of the application itself is outside of the domain 
%of the language.  

\paragraph{Flexibility} Existing high-level languages rely on 
fixed compilation strategies in mapping the high-level network 
program to a switch-level implementation.  Cocoon allows the 
programmer to specify how each high-level component is implemented 
and deployed, while automatically verifying the correctness of the 
implementation.

%Existing high-level SDN languages rely on fixed compilation 
%strategies in mapping the high-level network program to a 
%switch-level implementation.  Whenever the mapping produced by the 
%compiler does not meet the network operator's requirements, 
%compiler modifications are needed.  In contrast, Cocoon does not 
%enforce any particular compilation strategy, giving the programmer 
%complete control over the mapping process.


\vspace{2mm}

%Cocoon achieves these objectives via a programming model based on 
%\emph{iterative refinement}.  The programmer starts with a 
%high-level functional description of the desired network behavior, 
%focusing solely on the service that the network should provide to 
%each packet as opposed to how this service is implemented within 
%the network fabric.  This high-level description defines 
%correctness criteria that the resulting implementation must 
%satisfy.
%
%The developer iteratively refines the top-level specification 
%adding the details of topology, routing, fault recovery, etc.  The 
%final Cocoon specification contains sufficient detail to directly 
%generate an SDN application that manages network switches via a 
%Southbound API such as P4 or OpenFlow.  

Refinements are performed either manually or automatically.  In 
the latter case, the user picks one of an extensible set of 
\emph{refinement tactics}.  A tactic can be as simple as 
instantiating shortest-path routing within a segment of the 
network, or as complicated as the global NetKAT compilation 
algorithm~\cite{Smolka_EFG_15}.  The user is free to apply 
different tactics to different parts of the network.  Whenever no 
existing tactic matches application requirements, the user can 
implement their own custom solution via a manual refinement.

Refinements are performed in a modular way, with every refinement 
confined to a single component of the network.  Modularity is 
enforced at the language level: a Cocoon program defines a number 
of \emph{roles}, where each role models a group of similar 
entities, e.g., top-of-rack switches, virtual network function 
instances, or a segment of the switching fabric.  A refinement 
replaces one or more roles with a more detailed implementation, 
possibly splitting them into multiple roles.  

The modular refinement process facilitates clean separation of 
concerns and ensures that each individual refinement is amenable 
to automatic formal verification.  The Cocoon verifier proves for 
each refinement that the refined specification is functionally 
equivalent to the role it refines.  Correctness of each refinement 
in the chain guarantees that the final specification is 
functionally equivalent to the top-level specification.  

%The control plane is specified using a purely functional language 
%that 

In this talk, we will present the Cocoon's design philosophy, 
outline its syntax and semantics, as well as report on several 
case studies where we used Cocoon to synthesize and verify a range 
of SDN applications, including network virtualization, service 
chaining, a B4-style WAN~\cite{Jain_KMOPSVWZZZHSV_13}, and a 
software-defined Internet exchange~\cite{Gupta_MBCFRV_16}. 

\bibliographystyle{abbrv}

\setlength{\bibsep}{1pt}
%
\renewcommand{\bibfont}{\footnotesize}

\bibliography{refs}

\end{document}
