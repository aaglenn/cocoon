\documentclass{report}

\usepackage{hyperref}
\usepackage{color}
\usepackage[usenames]{xcolor}
\usepackage{tikz}
\usepackage{amsmath,amssymb}

\usepackage{algorithm}
\usepackage{algpseudocode}
\usepackage{stmaryrd}
\newcommand\sem[1]{\llbracket #1\rrbracket}
\newcommand{\pkt}{\ensuremath{pkt}}
\newcommand{\env}{\ensuremath{env}}
\newcommand{\args}{\ensuremath{args}}
\newcommand{\true}{\src{true}}
\newcommand{\false}{\src{false}}
\newcommand{\ERROR}{\src{ERROR}}


\usepackage[letterpaper, left=25mm, right=25mm]{geometry} 

% New definitions
\algnewcommand\algorithmicswitch{\textbf{case}}
\algnewcommand\algorithmiccase{\textbf{}}
% New "environments"
\algdef{SE}[SWITCH]{Switch}{EndSwitch}[1]{\algorithmicswitch\ #1\ {\bf of}}{\algorithmicend\ \algorithmicswitch}%
\algdef{SE}[CASE]{Case}{EndCase}[1]{\algorithmiccase\ #1 :}{\algorithmicend\ \algorithmiccase}%
\algtext*{EndSwitch}%
\algtext*{EndCase}%

%\usepackage[T1]{fontenc}

\definecolor{lgray}{gray}{0.9}
\definecolor{lyellow}{cmyk}{0,0,0.3,0}

\usepackage{listings}

\lstnewenvironment{ccnlisting}[1]
{\vspace{3mm}
 \lstset{
    backgroundcolor=\color{lyellow},
    basicstyle=\small\ttfamily, 
%    keywordstyle=\bfseries,
    keywordstyle=\underbar,
    identifierstyle=,
    commentstyle=\slshape,
    stringstyle=,
    showstringspaces=false,
    keywords={and,
              any,
              assume,
              bit,
              bool,
              check,
              default,
              drop,
              else,
              false,
              foreign,
              fork,
              function,
              if,
              in,
              int,
              key,
              match,
              sink,
              not,
              or,
              pkt,
              primary,
              procedure,
              references,
              refine,
              role,
              send,
              state,
              string,
              switch,
              switch_port,
              table,
              true,
              typedef,
              unique,
              var,
              view,
              the},
    sensitive=false,
    morecomment=[l]{//},
    morecomment=[s]{/*}{*/},
    numberstyle=\tiny,
    stepnumber=1,
    numbersep=1pt,
    emphstyle=\bfseries,
    belowskip=0pt,
    aboveskip=0pt,
    #1
}}{\vspace{3mm}}

\lstnewenvironment{bnflisting}[1]
{\vspace{3mm}
 \lstset{
    backgroundcolor=\color{lgray},
    basicstyle=\small\ttfamily,
    keywordstyle=\underbar,
    identifierstyle=,
    commentstyle=\slshape,
    stringstyle=,
    showstringspaces=false,
    keywords=,
    sensitive=false,
    morecomment=[l]{//},
    morecomment=[s]{/*}{*/},
    numberstyle=\tiny,
    stepnumber=1,
    numbersep=1pt,
    emphstyle=\bfseries,
    belowskip=0pt,
    aboveskip=0pt,
    #1
}}{\vspace{3mm}}


\newcommand{\src}[1]{\texttt{#1}}

\newcommand{\comment}[1]{{\textit{\textbf{#1}}}}


\title{Cocoon~2 Language Reference}

\begin{document}

\maketitle

\tableofcontents

\chapter{Syntax reference}\label{s:reference}

\section{Notation}

We use the "-list" notation to denote comma-separated lists of
syntactic entities, for example:

\begin{bnflisting}{}
<expression-list> := <expression> | <expression> "," <expression-list>
\end{bnflisting}

Cocoon is a case-sensitive language.  Furthermore, we use capitalization 
to disambiguate parsing.  In particular, constructor, relation, and role names must 
start with an upper-case letter; variable, field, column, function, and procedure names 
start with a lower-case letter; type names can start with either:

\begin{bnflisting}{}
<ucIdentifier> := [A-Z][a-zA-Z_0-9]*      -- upper-case
<lcIdentifier> := [a-z_][a-zA-Z_0-9]*     -- lower-case
<identifier>   := [A-Za-z_][a-zA-Z_0-9]*  -- upper or lower
<constructorName> := <ucIdentifier>
<relationName>    := <ucIdentifier>
<roleName>        := <ucIdentifier>
<functionName>    := <lcIdentifier>
<varName>         := <lcIdentifier>
<fieldName>       := <lcIdentifier>
<typeName>        := <identifier>
\end{bnflisting}



\section{Top-level declarations}

A Cocoon specification is a sequence of refinements:
\begin{bnflisting}{}
<spec> := <refinement>*
\end{bnflisting}

A refinement declaration starts with a possibly empty list of names of roles that 
are being refined.  These roles must be defined in the body of the refinement.
The body of the refinement consists of type definitions, relation
declarations, function and procedure declarations, 
role declarations, and assumptions.
\begin{bnflisting}{}
<refinement> := "refine" [<roleName-list>] 
                "{"<decl>*"}"

<decl> := <typeDef>
        | <relation>
        | <funcDecl>
        | <procDecl>
        | <roleDecl>
        | <assumption>
\end{bnflisting}

\subsection*{Example}

\begin{ccnlisting}{}
/* This refinement provides a new implementation of the LogicalPortIn role.
   It can also define new roles; however it cannot re-define other 
   previously defined roles. */
refine LogicalPortIn { 
    /* Type defintions */
    typedef port_id_t  = string
    typedef ip4_addr_t = bit<32>
    typedef ip6_addr_t = bit<128>
    typedef ip_addr_t  = IPAddr4 {addr4 : ip4_addr_t}
                       | IPAddr6 {addr6 : ip6_addr_t}
    typedef port_type_t = PortRegular  
                        | PortRouter   { rport: port_id_t}
                        | PortLocalnet { network_name: string}

    /* Uninterpreted function */
    function ethUnicastAddr(addr: mac_addr_t): bool
    
    /* Interpreted function */
    function isRouterPort(t: port_type_t): bool = 
        match (t) {
            PortRouter{_} -> true,
            _             -> false
        }

    /* Procedure: unlike functions, which are pure, procedures can
     * have side effects */
    procedure sourcePort(port: LogicalPort): LogicalPort =
        // trunk port?
        the (tport in TrunkPort | tport.port == port.name) {
            if (pkt.vlan == VLANNone) drop;
            var sport = the (x in LogicalPort 
                             | x.nested == NestedPort{port.name, pkt.vlan.vid}) x;
            pkt.vlan = VLANNone;
            sport
        } default {
            // drop packets the VLAN tags or multicast Ethernet source addresses
            if (pkt.vlan != VLANNone) drop;
            port
        }

    /* The following illustrates three types of relations:
       1. A regular table (updated by the controller, switches can only read)
       2. A computed view
       3. A state table (can be updated by switches and the controller) */
    table LogicalPort( 
           name      : port_id_t
         , switch_id : uuid_t
         , type      : port_type_t
         , nested    : nested_port_t
         , enabled   : bool
         , primary key (name)
         , foreign key (switch_id)         references LogicalSwitch(uuid)
         , foreign key (type.rport)        references LogicalRouterPort(name)
         , foreign key (type.network_name) references LocalNetwork(name)
         , foreign key (nested.parent)     references LogicalPort(name)
         , unique (nested.parent, nested.tag)
         , check (is_nested(nested) => not isRouterPort(type)))

    view ParentPort( child_name    : port_id_t
                   , child_switch  : uuid_t
                   , child_type    : port_type_t
                   , parent_name   : port_id_t
                   , parent_switch : uuid_t
                   , parent_type   : port_type_t
                   , parent_nested : nested_port_t
                   , check child_type == parent_type
                   , check child_switch == parent_switch
                   , check parent_nested == NestedPortNone{})
    /* Datalog rules to compute the ParentPort relation */
    ParentPort(cn: port_id_t, cs: uuid_t, ct, pn, ps, pt, pnest) :- 
        LogicalPort(cn, cs, ct, NestedPort{pn, _}, _),
        LogicalPort(pn, ps, pt, pnest, _)

    state table CTState ( port: port_id_t
                        , ctst: ct_state_t
                        , primary key (port))
    
    role LogicalPortIn[port in LogicalPort | iLogicalPort(port)] = {
        // drop all packets if port is disabled
        if (not port.enabled) drop;
        // drop multicast packets
        if (not ethUnicastAddr(pkt.src)) drop;
        var sport = sourcePort(port);
        // port security enabled? - check valid eth.src address
        the (p1 in LogicalPortSecured | p1.port == sport.name) {
            the (p2 in LogicalPortSecurity 
                 | p2.port==sport.name and p2.mac == pkt.src) {
                ()
            } default {
                drop
            }
        } default ();
        ...
    }
    ...
}
\end{ccnlisting}
    
\section{Types}

Type definition introduces a new user-defined type.  This type is visible in all
subsequent refinements.

\begin{bnflisting}{}
<typeDef> := "typedef" <typeName> ["=" <typeSpec>]
\end{bnflisting}

Omitting the \src{<typeSpec>} declares an \emph{opaque} type
that can only be manipulated by external functions.

\begin{bnflisting}{}
<typeSpec> := <arrayType>
            | <bitType>
            | <intType>
            | <stringType>
            | <boolType>
            | <structType>
            | <userType>
            | <tupleType>
\end{bnflisting}

\begin{bnflisting}{}
<bitType>    := "bit" "<" <decimal> ">"            // bit-vector
<intType>    := "int"                              // mathematical integer
<stringType> := "string"
<boolType>   := "bool"
<userType>   := <typeName>                         // type alias
<arrType>    := "[" <typeSpec> ";" <decimal> "]"   // fixed-size array
<structType> := <constructor> ("|" <constructor>)*
<tupleType>  := "(" <typeSpec-list> ")"            // tuple

<constructor> := <constructorName>                 // constructor without arguments
               | <constructorName> "{"<arg-list>"}"// constructor with arguments
<arg> := <varName> ":" <typeSpec>
\end{bnflisting}

Note:
\begin{itemize}
    \item Every complete Cocoon specification must declare \src{Packet} type.
        It must match packet headers supported by the target backend.
\end{itemize}

\subsection*{Example}

\begin{ccnlisting}{}
typedef port_id_t  = string
typedef ip6_addr_t = bit<128>
// type with multiple constructors
typedef ip_addr_t  = IPAddr4 {addr4 : ip4_addr_t}
                   | IPAddr6 {addr6 : ip6_addr_t}
// type with a single constructor
typedef eth_pkt_t = EtherPacket { src  : mac_addr_t
                                , dst  : mac_addr_t
                                , vlan : vlan_t
                                , l3   : l3_pkt_t}
typedef Packet = eth_pkt_t
typedef PIP = [bool; 3] // array of 3 booleans
\end{ccnlisting}

\section{Relations}

Relations are similar to database tables and are defined using SQL-like
syntax. The \src{table} keyword declares a relation that can only
be modified by the user or the controller; roles can only read such
relations.  The \src{state table} keyword defines a mutable relation 
that can be read and modified by roles.  The \src{view}
keyword declares a relation that is dynamically recomputed based on
other relations.  Views are defined using Datalog.  All three types of
views support standard SQL constraints: \src{primary key},
\src{foreign key}, \src{unique}, and \src{check}. 

\begin{bnflisting}{}
<relation> := ("state table" | "table" | "view") <relationName> 
             "(" <colOrConstraint-list> ")"
             <rule>* 
<colOrConstraint> := <column> | <constraint>
<column> := <fieldName> ":" <typeSpec>
<constraint> := "primary key" "(" <expression-list> ")"
              | "foreign key"  "(" <expression-list> ")" 
                         "references" <relationName> "(" <expression-list> ")"
              | "unique" "(" <expression-list> ")"
              | "check" <expression>
<rule> := <relationName> "(" <expression-list> ")" 
          ":-" <pattern-list> 
\end{bnflisting}

Note:
\begin{itemize}
    \item Both local and remote keys can refer to entire columns or 
          individual fields, e.g., the foreign key in the following 
          example refers to the \src{parent} field of the 
          \src{nested} column:
\begin{ccnlisting}{}
typedef nested_port_t = NestedPortNone
                      | NestedPort {parent: string, tag: vlan_id_t}
table LogicalPort(nested    : nested_port_t,
                  ...
                  foreign key (nested.parent) references LogicalPort(name),
                  ...)
\end{ccnlisting}
          This is interpreted as follows: if the value in the nested
          column is created using the \src{NestedPort} constructor,
          then its \src{parent} argument must refer to the \src{name}
          column of the \src{LogicalPort} relation.
    \item Patterns are discussed in Section~\ref{s:pattern} below.
    \item All expressions involved in relation declaration must be side-effect-free.
\end{itemize}



\section{Functions and procedures}

Functions are pure computations whose result depends only on their arguments.  
In particular, functions cannot access relations, send or drop packets, read 
or modify the content of the \src{pkt} variable, etc.  In contrast, 
procedures can contain arbitrary code.  Functions (but not procedures) can
be declared without a body.  Such functions can be used to invoke native code 
or to delay function definition to subsequent refinements or to run time.  
Undefined functions and treated as uninterpreted functions by the verifier.

\begin{bnflisting}{}
<funcDecl> := "function" <functionName> "(" [<arg-list>] ")" ":" <typeSpec>
                         ["=" <expression>] // optional function body
<procDecl> := "procedure" <functionName> "(" [<arg-list>] ")" ":" <typeSpec>
                         "=" <expression>   // mandatory procedure body
<arg> := <varName> ":" <typeSpec>
\end{bnflisting}

\section{Assumptions}

An assumption is a boolean expression that must be true for all values of variables 
that occur in the expression.  Assumptions are used to constrain the content of
relations (tables and views, but not state tables) and undefined functions.  

\begin{bnflisting}{}
<assumption> := "assume" <expression>
\end{bnflisting}

\subsection*{Example}

\begin{ccnlisting}{}
assume (LogicalPort(p,_,PortRouter{rport},_,_) and 
        LogicalRouterPort(rport,_,_,_,_,ppeer))
       => ppeer == PeerSwitch
\end{ccnlisting}

\section{Roles}

The main building blocks of Cocoon specifications are roles, 
which specify arbitrary network entities:
ports, hosts, switches, routers, etc. A role accepts a packet,
possibly modifies it and forwards to zero or more other
roles. Roles are parameterized, so a single role can specify a 
set of similar entities, allowing a large network to
be modeled with a few roles.  An instance of the role corresponds 
to a concrete parameter assignment, with the value of the parameter
ranging over a relation:
\begin{bnflisting}{}
<role> := "role" <roleName> "(" <varName>           // parameter
                                "in" <relationName> // relation
                                 ["|" <expr>]       // parameter filter
                                 ["/" <expr>]")"    // packet filter
          "=" <expr>                        // the body of the role
\end{bnflisting}
where the optional parameter filter restricts the set of instances to a subset
of the relation and the optional packet filter restricts the set of packets accepted 
by the role.  The parameter filter must be a side-effect-free boolean 
expression that only depends on the value of the parameter.  The packet
filter is a side-effect-free boolean expression that depends on parameter
and the \src{pkt} variable


\subsection*{Example}

\begin{ccnlisting}{}
role LogicalPortIn[port in LogicalPort | iLogicalPort(port)] = 
    if (not port.enabled) drop;
    ...
\end{ccnlisting}

\section{Patterns}\label{s:pattern}

A pattern describes a structural template that a value can be 
matched against. For example, the following pattern matches a 
value created with the \src{Constr1} constructor, whose second
argument was created with \src{Constr2}, and whose third 
argument is a 3-tuple.  
\begin{ccnlisting}{}
Constr1{_, Constr2{x,_}, (_,_,y)}
\end{ccnlisting}

\begin{bnflisting}{}
<pattern> := <varName>                                -- variable name
             "(" <pattern-list> ")"                   -- tuple
             "_"                                      -- placeholder
             <constructorName> "{" <pattern-list> "}" -- constructor
\end{bnflisting}

Patterns can occur in three contexts (see Section~\ref{s:expr} for more detail): 
\begin{enumerate}
    \item Left-hand side of an assignment:
\begin{ccnlisting}{}
(_, cts) = ct_track_from(ct_state_new(), pkt)
\end{ccnlisting}

    \item Arguments of a relation:
\begin{ccnlisting}{}
ParentPort(cn, cs, ct, pn, ps, pt, pnest) :- 
    LogicalPort(cn, cs, ct, NestedPort{pn, _}, _),
    LogicalPort(pn, ps, pt, pnest, _)
\end{ccnlisting}

    \item Match expressions:
\begin{ccnlisting}{}
match (port.nested) {
    NestedPortNone     -> ...,
    NestedPort{_, tag} -> ... 
}
\end{ccnlisting}
\end{enumerate}

\section{Expressions}\label{s:expr}

Cocoon is an expression-oriented language.  In particular, 
sequential composition, assignment, sending of a packet, etc., are 
all expressions.  The type of every expression is known at 
compile time.  Expressions that do not produce any meaningful 
results return an empty tuple.

\begin{bnflisting}{}
<expression> := 
          <term>
        | "not" <expression>
        | <expression> "%" <expression>
        | <expression> "+" <expression>
        | <expression> "-" <expression>
        | <expression> ">>" <expression>
        | <expression> "<<" <expression>
        | <expression> "++" <expression>
        | <expression> "==" <expression>
        | <expression> "!=" <expression>
        | <expression> "<" <expression>
        | <expression> "<=" <expression>
        | <expression> ">" <expression>
        | <expression> ">=" <expression>
        | <expression> "and" <expression>
        | <expression> "or" <expression>
        | <expression> "=>" <expression>             // implication
        | <expression> "." <fieldName>               // struct field
        | <expression> "["<decimal> "," <decimal>"]" // bit slice
        | <expression> ":" <typeSpec>                // specify type of expression
        | <expression> "=" <expression>              // assignment
        | <expression> ";" <expression>              // sequence
        | <expression> "|" <expression>              // fork


<term> := "(" <expression> ")"
        | "{" <expression> "}"
        | <structTerm>   // struct given by its constructor
        | <applyTerm>    // function or procedure call
        | <predTerm>     // relational predicate 
        | <intTerm>      // integer constant
        | <boolTerm>     // boolean constant
        | <stringTerm>   // string literal
        | <packetTerm>   // special pkt variable
        | <varName>      // variable reference: role key or local var
        | <matchTerm>    // match expression
        | <dropTerm>     // drop the packet
        | <iteTerm>      // if-then-else
        | <locTerm>      // location term (used in send expression)
        | <sendTerm>     // send expression
        | <varDeclTerm>  // variable declaration
        | <forkTerm>     // fork-expression
        | <theTerm>      // the-expression
        | <anyTerm>      // any-expression
\end{bnflisting}

\begin{bnflisting}{}
// struct term: constructor name followed by a list of arguments
<structTerm>   := <constructorName> "{" <expression-list> "}"
// function or procedure name followed by a list of arguments
<applyTerm>    := <functionName> "(" [<expression-list>] ")"
// relational predicate: relation name with the list of arguments,
// evaluates to true iff there exists an assignment to placeholders that
// satisfies the relation
<predTerm>     := <relationName> "(" <pattern-list> ")"
<intTerm>      := <decimal>
                  [<width>] "'d" <decimal>
                | [<width>] "'h" <hexadecimal>
                | [<width>] "'o" <octal>
                | [<width>] "'b" <binary>
<width> := <decimal>
<boolTerm>     := "true" | "false"
<stringTerm>   // - parses a quoted string literal; 
               //   deals correctly with escape sequences and gaps
<packetTerm>   := "pkt"
<matchTerm>    := "match" "(" <expression> ")" "{"
                       <match-case-list>
                  "}"
<match-case>   := <pattern> "->" <expression>
<dropTerm>     := "drop"
<iteTerm>      := "if" <term> <term> ["else" <term>]
<locTerm>      := <roleName> "[" <expression> "]"
<sendTerm>     := "send" <expression>
<varDeclTerm>  := "var" <varName>
<forkTerm>     := "fork" "("<varName> "in" <relationName> ["|"<expression>]")" 
                         <term>
<theTerm>      := "the" "("<varName> "in" <relationName> ["|"<expression>]")" 
                        <term>
                        ["default" <term>]
<anyTerm>      := "any" "(" <varName> "in" <relationName> ["|" <expression>] ")" 
                   <term>
                   ["default" <term>]
\end{bnflisting}

Note:
\begin{itemize}
    \item Relational predicates appear in Datalog rules and in assumptions.
        A predicate is satisfied if there exists an assignment to all 
        placeholders occurring in all its patterns.  Evaluating the predicate
        binds all variables that occur in it.
    \item Cocoon performs small amount of type inference.  For example, it
        inferes types of all variables declared in a pattern.  When type 
        inference is not possible or the user wants to make type of an expression
        explicit, the \src{<expression> : <typeSpec>} syntax is used.
    \item The \src{fork (var in rel | cond) body} expression executes \src{body}
        for each assignement of \src{var} to an element of relation \src{rel}
        that satisfies \src{cond}.  It returns 
    \item The \src{the (var in rel | cond) e1 default e2} expression executes 
        selects a unique assignment to \src{var} from \src{rel} that satisfies
        \src{cond} and evaluates \src{body} on it.  If there are no satisfying 
        assignments, then the \src{default} branch is executed.  It is an error 
        if more than one satisfying assignments exist.  Likewise, it is an error
        if there no elements of the relation satisfies the condition, and the 
        programmer has not supplied a default branch.  Both types of errors should
        be detected statically.
    \item The \src{any} construct is similar to the \src{the} construct, except
        that it non-deterministically selects one satisfying element from the relation,
        if there are more than one.  Thus, it can be used to model underspecified
        behaviors.
\end{itemize}

\end{document}
